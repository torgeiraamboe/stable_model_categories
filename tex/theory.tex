

\section{Definition}

Before we define stable model categories and explore their properties, 
we recall some definitions and facts. 

\begin{definition}{Pointed category}

We call a category $\mathcal{C}$ pointed if it has both an initial and a terminal object which coincide. 
This object is often called the zero object, 
and naturally denoted by $0$. 
\end{definition}

In the previous lecture we defined the suspension of an object $X$ in a model category $\mathcal{C}$ to be the homotopy pushout of the diagram $0\longleftarrow X \longrightarrow 0$, 
i.e. 
\begin{center}
\begin{tikzcd}
X \arrow[r] \arrow[d] & 0 \arrow[d] \\
0 \arrow[r]           & \Sigma X   
\end{tikzcd}
\end{center}

The assignment $X\longrightarrow \Sigma X$ almost defines an endofunctor $\Sigma : \mathcal{C}\longrightarrow \mathcal{C}$ naturally referred to as the suspension functor. 
It is a functor on the homotopy category of the model category, 
which we previously defined to be the localization of the model category at the weak equivalences. 
Since we use the homotopy pushout to define it, 
we only have functoriality up to weak equivalence in the model category itself, 
but this reduces to actual functoriality on the homotopy category. 

\begin{definition}{Stable model category}

A model category $\mathcal{M}$ is called stable if it is pointed, 
and the endofunctor $\Sigma$ is an autoequivalence on the homotopy category $M[Q^{-1}]$. 
\end{definition}

After seeing a definition it is important to see some examples. 
There are two important motivating examples of stable model categories, 
and luckily one is from algebra and one is from topology. 

The algebraic example is the category of unbounded complexes of modules over some ring. 
Here the suspension functor is just the shift functor, 
which we know is invertible and hence an autoequivalence. 
In a previous lecture we also (almost) defined a model structure on this category. 
We defined it for chain complexes in non-negative degrees, 
but the idea is the same. We define the weak equivalences to be the quasi-isomorphisms, 
the fibrations to be the surjections, 
and induce the cofibrations to be the ones that have the lifting property with respect to surjective quasi-isomorphisms. 
We also know that the complex 
\begin{equation*}
    \cdots \longrightarrow 0 \longrightarrow 0 \longrightarrow 0 \longrightarrow \cdots 
\end{equation*}
is a zero object in that category. 
Hence it is a stable model category. 
It's homotopy category is the derived category of $R$, 
i.e. $D(R)$. 
This category we know is triangulated from the homological algebra course. 
Even more generally we can do this for any additive category. 

\section{Spectra}

The topological example is (i think) for most people less familiar. 
To introduce it we need some machinery from an earlier lecture. 

Recall that Brown representability makes sure that certain nice cohomological functors are representable. 
In the category of topological spaces, 
there are several different notions of cohomology, 
even generalized versions. 
For those who have had the course MA3403 - Algebraic Topology 1, 
we know that a cohomology theory is required to satisfy the five Eilenberg-Steenrod axioms. 
\begin{itemize}
    \item Homotopy
    \item Excision
    \item Dimension
    \item Additivity
    \item Exactness
\end{itemize}

Removing the axiom of dimensionality, 
we get what are called generalized cohomology theories. 
These are often a bit more hard to explain and use, 
but the most famous example is topological K-theory.

In the algebraic case this is done by letting the new objects be chain complexes instead of just objects i the additive category. 
This is a generalization as we can interpret any one object as a chain complex centered in degree zero. 
So why did we do this?

Homoloical algebra is very much about (co)homology, 
and passing to chain complexes instead of single objects allows us to do all (co)homology constructions using a single object, 
i.e. defining the n'th (co)homology to be cycles modulo boundaries in degree $n$. 
The way we use (co)homology in algebraic topology is to assign a chain complex to a space and apply this (co)homology construction to it. 


Anyways... 
These generalized cohomology theories have their own set of cohomology functors, 
which are functors $\mathcal{E}^n:Top \longrightarrow Grp$, 
and it turns out that all these satisfy the niceness conditions required to use Brown representability on the homotopy category of CW complexes. 
This means that we can represent them by objects in $HoCW$. 
Thus we have for every $n$ and every generalized cohomology theory $\mathcal{E}$ a topological space $E_n$ such that $\mathcal{E}^n = [-, E_n]$. 

The collection of the spaces $E = \{E_n\}$ together with maps $\Sigma E_n \longrightarrow E_{n+1}$ is called a spectrum. 


\begin{example}{Suspension spectrum}

Take any topological space $T$ and let $E_0 = T$. 
We then set $E_n = \Sigma^n T$, and the maps to be isomorphisms. 
\end{example}

\begin{example}{The sphere spectrum}
    
    Let $E_n = S^n$ and the maps $\Sigma S^n \longrightarrow S^{n+1}$ be the canonical homeomorphisms. 
    Then we call $\mathbb{S}=\{S^n\}$ the sphere spectrum. 
    It is also the suspension spectrum of $S^0$.
\end{example}

The sphere spectrum represents cohomotopy as a cohomology theory. 
The corresponding homology theory is stable homotopy theory. 

\begin{example}{The Eilenberg-MacLane spectrum}
    
    Say we want to compute ordinary singular cohomology. 
    This theory is as we now know represented by a spectrum. 
    We then have $H^n(X;G) = [X, K(G, n)]$, for some space $K(G, n)$. 
    These spaces are called the n-th Eilenberg Maclane space og $G$ and is a space such that 
    \begin{equation*}
        \pi_m(K(G, n)) = 
        \begin{cases}
            0, \quad m\neq n \\
            G, \quad m=n
        \end{cases}
    \end{equation*}
    All these Eilenberg-MacLane spaces neatly form a spectrum called the Eilenberg-MacLane spectrum. 
    For more information, take the course MA3408 - Algebraic topology 2.
\end{example}

We see that we have many spectra, 
and as all things they form a category. 
The maps are maps between all the topological spaces, 
such that they commute with the maps inside the spectra. 
We can kind of think about spectra as being the closest thing we have to chain complexes in topology, 
and that intuition can get you quite far. 

In almost the same way as before, 
the suspension functor is built into the objects, 
and applying the functor merely ``shifts'' the spectrum to the left, 
same as for chain complexes. 
This is at least very clear using suspension spectra. 
We can also shift back, 
which coincide on the homotopy category of spectra. 

We have not talked about the model structure on spectra, 
but it comes from the model structure we already had on topological spaces, 
just in some sense stabilized. 
For the people familiar with the theory, 
we can stabilize the induced model structure from topological spaces by Bousfield localizing at the homology theory $\pi_\ast(-)$. 

\subsection*{Ring spectra}

Often we really want to collect the (co)homological information together, 
which often is done by studying the graded homology group, 
or even better the graded cohomology ring of a space. 
This is a total invariant. 
If we want to represent a functor that lands in graded abelian groups, 
we need a graded object of topological spaces. 
This is where the spectra comes in. 

The total cohomology should then not be understood as a collection of abelian groups, 
or for generalized cohomology, 
$\mathcal{E}^* $-modules, 
where $\mathcal{E}^*$ is the cohomology of a point. 
It should be understood as a graded ring, 
or a graded $\mathcal{E}^* $-algebra. 
This means that we need a graded $\mathcal{E}^*$-algebra in topological spaces to represent the total cohomology functor. 
These objects are the so called ring spectra in topological spaces and are the correct objects for representing graded cohomology algebras of spaces. 

\begin{theorem}

Let $M$ be a stable model category with a compact generator $P$. 
Then $M$ is equivalent to a category of modules over a ring spectrum.

More precisely there is a one object $Sp$-category $C$ with $A\simeq End_M(P)$ an $A_{\infty}$-algebra such that $M\simeq AMod$
\end{theorem}

An $Sp$-category is kind of a non-linear dg-category.

%https://www.sciencedirect.com/science/article/pii/S004093830200006X

\begin{theorem}

Let $M$ be a stable model category (may need some niceness conditions). 
Then $M$ is equivalent to a category of presheaves of ring spectra. 
\end{theorem}

An example is the category of modules over the commutative ringspectrum $H\mathbb{Q}$, 
which is Quillen equivalent to the category of dg algebras over $\mathbb{Q}$. 


\section{Triangulation on the homotopy category}

The main reason to introduce anything with the prefix ``stable'' is seemingly to show that it is triangulated, 
so we will do just that. 
We have only seen this once so far in this seminar, 
and that was when stabilizing a Frobenius category, 
which made it triangulated. 
There we had some category, 
with some distinguished class of sequences, 
where we could do fibrant and cofibrant replacements by injective or projective objects. 
Since it was a Frobenius category, 
these two types of objects, 
i.e. the fibrant and the cofibrant ones, 
in fact coincided. 
This we know makes taking homotopy classes of maps equivalent to producing the derived category (Quillen localization). 
Hence we had some model category, 
where the suspension was an autoequivalence up to homotopy, 
which made the stabilization (really the homotopy category) triangulated. 

This idea continues, 
and is formalized through defining the following to be a triangle in a stable model category. 

\begin{center}
\begin{tikzcd}
    X \arrow[r] \arrow[d] & Y \arrow[r] \arrow[d] & 0 \arrow[d] \\
    0 \arrow[r]           & Z \arrow[r]           & \Sigma X   
\end{tikzcd}
\end{center}

Lets prove that this makes the homotopy category triangulated. 
All the squares here represent homotopy pullbacks and homotopy puchouts. 

\textbf{TR1:} The first axiom is satisfied due to the identity being a homotopy pullback of the zero object along $X$.  

\begin{center}
\begin{tikzcd}
    X \arrow[r] \arrow[d] & X \arrow[r] \arrow[d] & 0 \arrow[d] \\
    0 \arrow[r]           & 0 \arrow[r]           & \Sigma X   
\end{tikzcd}
\end{center}

The second part comes from iterated homotopy pushouts. 

\begin{center}
\begin{tikzcd}
    X \arrow[r] \arrow[d] & Y \arrow[r] \arrow[d] & 0 \arrow[d] \\
    0 \arrow[r]           & 0\coprod_X Y \arrow[r]           & (0\coprod_X Y)\coprod_Y 0   
\end{tikzcd}    
\end{center}

Since iterated homotopy pushout is again a homotopy pushput, 
we know that the space $ (0\coprod_X Y)\coprod_Y 0 $ must be homotopy equivalent to $\Sigma X$. 

\textbf{TR2:} The second axiom comes from attaching two zeroes to a distinguished triangle. 
Then we take the homotopy pushout, 
which must be homotopy equivalent to $\Sigma Y$.

\begin{center}
\begin{tikzcd}
    X \arrow[r] \arrow[d] & Y \arrow[r] \arrow[d] & 0 \arrow[d] \\
    0 \arrow[r]           & Y/X \arrow[r]\arrow[d] & \Sigma X \arrow[d] \\
                          & 0 \arrow[r]           & \Sigma Y 
\end{tikzcd}    
\end{center}

The $-1$ sign that appears in the shifted triangle occurs when ``transposing'' the ``vertical'' distinguished triangle

\begin{center}
\begin{tikzcd}
    Y \arrow[r] \arrow[d] & 0 \arrow[d] \\
    Y/X \arrow[r]\arrow[d] & \Sigma X \arrow[d] \\
    0 \arrow[r]           & \Sigma Y 
\end{tikzcd}    
\end{center}

into a ``horizontal'' one to get it to the proper form. 


\textbf{TR3:} I unfortunately couldn't quite find a nice and easy diagram argument for this axiom. 
But, it will be proven next time using stable derivators. 


\textbf{TR4:} The fourth and final axiom is the octahedral axiom. 
To prove it we must show that the composition of two triangles again form a triangle nicely. 
As usual we attach some zeroes to the composition diagram, 
and take homotopy pushouts until we reach a suitable end. 
By inspecting the iterated homotopy pushouts, 
we see that we indeed get the following diagram.  

\begin{center}
\begin{tikzcd}
X \arrow[r] \arrow[d] & Y \arrow[r] \arrow[d]   & Z \arrow[d] \arrow[r]   & 0 \arrow[d]                  &             \\
0 \arrow[r]           & Y/X \arrow[r] \arrow[d] & Z/X \arrow[r] \arrow[d] & \Sigma X \arrow[d] \arrow[r] & 0 \arrow[d] \\
                      & 0 \arrow[r]             & Z/Y \arrow[r]           & \Sigma Y \arrow[r]           & \Sigma Y/X 
\end{tikzcd}   
\end{center}

Note that this proof was done in a bit more detail and explanation in the actual presentation. 
I'm sorry for this proof being short an detail lacking, 
but it must be so due to time constraints.. 